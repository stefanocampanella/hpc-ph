\subsubsection{Sommario}\label{sommario}

In questo documento sono raccolti gli appunti e le sbobinature del corso
\emph{``Calcolo ad Alte Prestazioni per la Fisica''}, tenuto dal
professor Giacinto Donvito al dipartimento interateneo di Fisica
dell'Università degli studi di Bari nell'anno accademico 2015/2016.

Per la parte di strumenti per il calcolo scientifico si rimanda al
manuale \emph{Amministrare GNU/Linux} di S. Piccardi, distribuito con
licenza GNU Free Documentation License (FDL) all'indirizzo:

\url{https://labs.truelite.it/attachments/download/821/corso.pdf}

ed in particolare, con riferimento alla revisione 1367 del 9/07/2014, ai
capitoli 1, 2, 6 ed i paragrafi 4.1, 4.2, 4.3, 5.1 (ed eventualmente del
8.3).

Una utile introduzione sull'utilizzo di \texttt{make} si trova nella
appendice ``Gli strumenti di ausilio per la programmazione'' di
\emph{Guida alla Programmazione in Linux} (GaPiL) sempre di S. Piccardi
e distribuita anch'essa con licenza FDL all'indirizzo

\url{http://gapil.gnulinux.it/files/2011/12/gapil.pdf}

\textbf{ATTENZIONE:} Questo documento viene distribuito nella speranza
possa essere utile, tuttavia non è garantita la correttezza dei
contenuti o della forma. Sono ben gradite segnalazioni di eventuali
errori.

\section{Teoria dei calcolatori e del calcolo
scientifico}\label{teoria-dei-calcolatori-e-del-calcolo-scientifico}

\subsection{Modello di Von Neumann}\label{modello-di-von-neumann}

Praticamente tutti i computer attuali fondano la propria architettura
sul \emph{modello di Von Neumann}, quest'ultimo si basa sul principio
secondo cui i dati e le istruzioni condividono lo stesso spazio di
memoria. Lo schema è il seguente, ci sono 5 unità fondamentali:

\begin{enumerate}
\def\labelenumi{\arabic{enumi}.}
\tightlist
\item
  \textbf{CPU (o unità di lavoro)} che si divide a sua volta in
\end{enumerate}

\begin{itemize}
\tightlist
\item
  \textbf{Unità di controllo}, che gestisce le operazioni necessarie per
  eseguire una istruzione o un insieme di istruzioni
\item
  \textbf{Unità operativa}, la cui parte più importante è l'unità
  aritmetico-logica (o ALU) e che effettua appunto operazioni
  aritmetiche e logiche
\end{itemize}

\begin{enumerate}
\def\labelenumi{\arabic{enumi}.}
\setcounter{enumi}{1}
\tightlist
\item
  \textbf{RAM (Random Access Memory)}: Unità di memoria, intesa come
  memoria di lavoro o memoria principale
\item
  \textbf{Unità di input} (tastiere, mouse, ecc.)
\item
  \textbf{Unità di output} (monitor, stampanti, plotter, ecc.)
\item
  \textbf{Bus} ovvero un canale che collega tutti i componenti fra loro
\end{enumerate}

In un calcolatore convenzionale le istruzioni vengono processate
\textbf{sequenzialmente} nei seguenti passi:

\begin{enumerate}
\def\labelenumi{\arabic{enumi}.}
\tightlist
\item
  un istruzione viene caricata dalla memoria (fetch) e decodificata
\item
  vengono calcolati gli indirizzi degli operandi
\item
  vengono prelevati gli operandi dalla memoria
\item
  viene eseguita l'istruzione
\item
  il risultato viene scritto in memoria (store)
\end{enumerate}

Esistono pertanto due strategie per migliorare le performance

\begin{itemize}
\tightlist
\item
  Aumentare le prestazioni dei singoli componenti elettronici
\item
  Eseguire più istruzioni contemporaneamente
\end{itemize}

Si precisa che in generale un piccolo aumento delle performance non è
interessante da un punto di vista tecnologico e pertanto si considera un
incremento significativo intorno ad un ordine di grandezza (fattore 10).

La prima strategia ha dei limiti fisici:

\begin{itemize}
\tightlist
\item
  Problema della dissipazione del calore
\item
  Limite imposto dalla velocità della luce
\end{itemize}

Il problema della dissipazione del calore è collegato alla dimensione
dei transistor che compongono la CPU, questa dimensione è a sua volta
collegata alle dimensioni caratteristiche del processo di litografia
utilizzato (feature dimension). Il limite attuale per queste dimensioni
è intorno ai 10 nanometri e non è possibile scendere molto al disotto di
questa dimensione per via di effetti quantistici (oltre naturalmente al
limite imposto dalle dimensioni atomiche del supporto di silicio).

L'altro limite è la velocità di I/O che è vincolata al limite fisico
della velocità della luce per i segnali elettromagnetici nel vuoto.

Altri problemi sono la dimensione della ram, che pone un limite alle
applicazioni in esecuzione (attualmente su un singolo slot 128GB), la
larghezza di banda tra memoria e processore, o fra processore/memoria e
I/O, la larghezza di banda della cache, ecc.

\subsection{Tassonomia di Flynn}\label{tassonomia-di-flynn}

La prima forma di parallelismo sperimentata è il pipelining e presenta
delle analogie con il concetto di catena di montaggio dove, in una linea
di flusso (pipe) di stazioni di assemblaggio gli elementi vengono
assemblati a flusso continuo.

In questo caso le componenti elettroniche specializzate (unità) che
operano su ciascuna delle 5 fasi attraverso cui viene eseguita una
istruzione (dal fetch allo store) funzionano simultaneamente. Ciascuna
lavora sulla istruzione seguente rispetto alla unità successiva e sulla
istruzione precedente rispetto alla unità precendente. In questo modo si
evitano i tempi morti sulle singole unità.

Idealmente tutte le stazioni di assemblaggio devono avere la stessa
velocità di elaborazione, altrimenti la stazione più lenta diventa il
\emph{bottleneck} della intera pipe.

La tassonomia di Flynn è una classificazione delle molteplicità
dell'hardware per manipolare stream di istruzioni e di dati. In
particolare lo \textbf{stream delle istruzioni} (sequenza delle
istruzioni eseguite dal calcolatore) può essere singolo, \textbf{SI} per
Single Instruction stream, o multiplo, \textbf{MI} per Multiple
Instruction stream, ed anche lo \textbf{stream di dati} (sequenza degli
operandi su cui vengono eseguite le istruzioni) può essere a sua volta
singolo, \textbf{SD} per Single Data stream, o multiplo, \textbf{MD} per
Multiple Data stream. Complessivamente sono possibili 4 combinazioni:

\begin{itemize}
\tightlist
\item
  \textbf{SISD}: è la tipica architettura di Von Newman (sistemi scalari
  monoprocessore) e può essere pipelined
\item
  \textbf{MISD}: più flussi di istruzioni lavorano contemporaneamente su
  un unico flusso di dati. Non è stata finora utilizzata praticamente.
\item
  \textbf{SIMD}: una singola istruzione opera simultaneamente su più
  dati, in questo caso il sistema possiede una singola unità di
  controllo ma diverse unità di elaborazione. Sistemi di questo genere
  vengono anche chiamati \emph{array processor} o \emph{sistemi
  vettoriali}. Anche questo genere di sistemi può essere pipelined.
  Esistono esempi famosi di supercomputer vettoriali (come quelli
  costruiti dalla Cray negli anni 70) e sviluppati per applicazioni
  particolari, oggi molti dei moderni processori hanno capacità SIMD ed
  un set di istruzioni dedicato (SSE, Streaming SIMD Extensions, nel
  caso Intel).
\item
  \textbf{MIMD}: in questo caso si parla anche di parallelismo asincrono
  ed esistono più CPU che operano su dati diversi, ovvero più unità di
  controllo ciascuna collegata a più unità operative. In questo senso ci
  si può riferire a questi sistemi come ad una versione multiprocessore
  dei sistemi SIMD.
\end{itemize}

Per realizzare questi sistemi è necessario che i vari sottosistemi
comunichino tra loro. Questo può avvenire sia in un solo calcolatore con
\textbf{un'unica memoria condivisa} fra più processori (per questo detti
\emph{tightly coupled}), che per questo devono trovarsi nello stesso
spazio fisico, sia tramite una rete di calcolatori (con processori in
questo caso \emph{loosely coupled}) interconnessi e funzionalmente
completi (dotati di CPU, RAM, bus, dischi, etc.), che possono anche
trovarsi in posti geograficamente differenti. Inoltre sono anche
possibili soluzioni combinate di questi due casi.

I sistemi di tipo MIMD abbracciano una ampia classe di architetture
possibili, ricapitolando si ha

\begin{itemize}
\tightlist
\item
  \textbf{Architettura MIMD multiprocessore o shared memory system}: i
  processori condividono i dati e le istruzioni in una memoria centrale
  comune. La comunicazione avviene dunque mediante condivisione della
  memoria attraverso un bus.
\item
  \textbf{Architettura MIMD multicomputer o distributed memory system}:
  ogni processore ha una memoria propria. I vari processori comunicano
  tra loro mediante una rete che consente a ciascun processore di
  trasmettere e ricevere dati dagli altri. I processori possono anche
  essere fisicamente lontani.
\end{itemize}

Inoltre il genere di calcoli che vengono eseguiti su un sistema
parallelo può essere di due tipi, distinti per il genere di priorità che
comportano e di performance richieste al sistema

\begin{itemize}
\tightlist
\item
  \textbf{HPC, High Performance Computing}: il calcolo HPC consiste
  nell'esecuzione di \emph{task} computazionalmente intensivi nel minor
  tempo possibile ed è dunque caratterizzato dalla necessità di grandi
  capacità di calcolo distribuite in brevi periodi di tempo (ore o
  giorni). Le performance di sistemi per HPC sono spesso misurate in
  \emph{FLOPS} (FLoating point OPerations per Second).
\item
  \textbf{HTC, High Throughput Computing}: il calcolo HTC consiste
  usualmente nell'esecuzione di molti task debolmente correlati (o di
  singoli task su grandi moli di dati) che si ha interesse siano
  completati efficientemente lungo periodo (mesi o anni).
\end{itemize}

La velocità della condivisione di istruzioni e dati caratteristici dei
sistemi a memoria condivisa contro la flessibilità e la scalabilità dei
sistemi a memoria distribuita tendono in maniera natuarale a far
identificare i primi come soluzioni per HPC ed i secondi per HTC.

Anche in questo caso (specialmente nel caso di grandi sistemi) questi
paradigmi possono combinarsi (ad esempio i sottosistemi di una
infrastruttura per HTC possono essere effettivamente sistemi HPC).

\subsection{SETI@home}\label{setihome}

SETI@home è stato un progetto per la ricerca di segnali radio
compatibili con segni di vita intelligente extraterrestre. Questo
progetto è stato tra i primi e più famosi ad essersi avvalsi di una
infrastruttura di calcolo distribuito \emph{volontario} basata su
internet, ovvero ad aver utilizzato le risorse, momentaneamente e
gratuitamente messe a disposizione, di personal computer geograficamente
distribuiti e connessi ad un nodo centrale tramite una connessione
internet.

Oggi parte della tecnologia sviluppata per SETI@home è confluita in una
infrastruttura per il calcolo distribuito indipendente dal particolare
progetto di ricerca denominata BOINC (Berkley Open Infrastructure for
Network Computing). Tramite quest'ultima gli utenti possono decidere di
donare il proprio tempo di CPU ad uno o più progetti. Lo stesso
SETI@home prosegue oggi come uno dei progetti ospitati da BOINC.

Nel caso di SETI@home l'obiettivo era analizzare tutto lo spettro delle
frequenze di un radio telescopio per l'intero tempo di osservazione. Ciò
che ha reso possibile l'impiego della infrastruttura di calcolo
descritta

\begin{enumerate}
\def\labelenumi{\arabic{enumi}.}
\tightlist
\item
  I dati possono essere spacchettati in piccole porzioni (dell'ordine di
  500 Kbyte), non è pertanto necessaria una connessione ad alte
  performance fra i singoli nodi.
\item
  Il tempo di CPU necessario per analizzare tali porzioni di dati su
  hardware di consumo è relativamente breve.
\item
  L'analisi dei segnali in ciascuna porzione è del tutto indipendente
  dalle altre ed è pertanto possibile eseguire tutti i job
  contemporaneamente (si dice che il problema è \emph{embarassingly
  parallel} o \emph{perfectly parallel}).
\end{enumerate}

\subsection{Cluster}\label{cluster}

Un generico gruppo di calcolatori interconnessi ed in grado di lavorare
cooperativamente come un unico sistema è in generale indicato come
\emph{cluster}.

Perchè un cluster possa essere considerato come un sistema a memoria
distribuita è necessario: - hardware di rete ad elevate prestazioni -
uno strato software che implementi le funzionalità richieste
(interfacce, protocolli, librerie, etc.) - applicativi che sfruttino le
capacità del sistema

Un esempio di queste librerie è \emph{MPI} (Message Passing Interface)
che permette di implementare in diversi linguaggi (C, C++, Fortran,
Python, Julia) applicativi per sistemi a memoria distribuita. L'ovvia
premessa è che, indipendentemente dal sistema particolare o dal
linguaggio di programmazione, l'algoritmo sia parallelizzabile.

In generale esistono tre tipi di cluster (i primi due sono i più
diffusi):

\begin{enumerate}
\def\labelenumi{\arabic{enumi}.}
\tightlist
\item
  \textbf{Fail-over Cluster}: il funzionamento delle macchine è
  continuamente monitorato e quando una di queste ha un malfunzionamento
  un'altra subentra in attività.
\item
  \textbf{Load Balancing Cluster}: è un sistema nel quale le richieste
  di lavoro sono inviate alla macchina con meno carico di elaborazione
  distribuendo/bilanciando così il carico di lavoro sulle singole
  macchine.
\item
  \textbf{HPC Cluster}: le macchine suddividono l'esecuzione di un
  \emph{job} in più processi e questi ultimi vengono istanziati
  simulataneamente su più macchine (calcolo distribuito).
\end{enumerate}

In generale i vantaggi offerti da un cluster rispetto ad un singolo
calcolatore sono

\begin{itemize}
\tightlist
\item
  \emph{Incremento delle capacità di calcolo} grazie allo sfruttamento
  di più unità di calcolo e maggiore disponibilità di memoria
\item
  \emph{Maggiore affidabilità} in quanto il sistema continua a
  funzionare anche in caso di guasti di parti di esso (ovvero dei
  singoli calcolatori)
\item
  \emph{Minori costi}, infatti questi sistemi sono fino a 15 volte più
  economici dei tradizionali supercomputer rispetto ai quali, a parità
  di prestazioni, permettono un notevole risparmio sui componenti
  hardware
\item
  \emph{Scalabilità hardware}, possibilità di aggiungere ho rimpiazzare
  singole componenti a caldo
\item
  \emph{Disponibilità di un gran numero di software Open Source} come
  MOSIX e openMosix.
\end{itemize}

Gli svantaggi principali sono:

\begin{itemize}
\tightlist
\item
  Difficoltà di gestione e di organizzazione di un elevato numero di
  calcolatori
\item
  Possibilità di problemi di connessione e di banda passante
  (specialmente di calcolatori lontani)
\item
  Assieme allo hardware scala anche la probabilità di failure
\item
  Scarse prestazioni nel caso di applicazioni non parallelizzabili
\end{itemize}

Fra i primi cluster si ricordano i cluster Beowulf, ovvero cluster di
semplici personal computer collegati tramite reti informatiche standard,
senza l'utilizzo di hardware esplicitamente sviluppato per il calcolo
parallelo. Cluster di questo genere erano in genere cluster di computer
IBM compatibili, implementati utilizzando software Open Source con lo
scopo di eseguire task computazionalmente intensivi in genere in ambito
tecnico scientifico.

I Beowulf cluster sono caratterizzati da - Accesso al sistema possibile
solo dal nodo principale, che spesso è l'unico ad essere dotato di
tastiera e monitor. - Nodi basati su di una architettura standard
(x86,AMD64,PowerPC,etc.), usualmente uguale per tutti i nodi. - Utilizzo
di un sistema operativo (in genere GNU/Linux) e software Open Source

\subsection{Batch System}\label{batch-system}

Gli atomi di calcolo (in cui viene istanziato un processo, letti o
scritti dati, etc.) in un sistema multiutente (come può essere un
cluster) vengono detti \textbf{job} (utilizzando una terminologia
propria dei sistemi UNIX) e la richiesta di esecuzione di un job da
parte di un utente è detta sottomissione di un job.

Per poter distribuire i job degli utenti su un cluster è necessario un
software chiamato batch system (o PBS, Portable Batch System). Questi
software si occupano di gestire l'ordine di esecuzione dei job e, in un
sistema distribuito, di scegliere il nodo su cui verranno eseguiti.
Dunque i batch sistem hanno il compito di accomodare le richieste degli
utenti (\emph{domanda}) e la disponibilità (limitata) di risorse
(\emph{offerta}), operazione nota in questo contesto come
\textbf{matchmaking}.

Alcuni esempi di batch system sono Condor, Torque/Maui, LSF, SLURM,
PBSPro, Oracle Grid Engine, Open Grid Engine.

La maggior parte di questi sistemi organizzano i job in una o più code e
per questa ragione sono anche noti come \emph{sistemi di code}.
L'operazione di ordinare i job nella coda di esecuzione è nota come
\textbf{job scheduling}, o scheduling, ed esistono diversi algoritmi che
tentano di ottimizzare il problema dello scheduling rispetto a diversi
parametri.

La premessa ovvia al problema dello scheduling è la capacità di
monitorare le risorse.

Genericamente i parametri rispetto a cui viene ottimizzato un algoritmo
di scheduling sono

\begin{itemize}
\tightlist
\item
  \textbf{Throughput}: il numero totale di job che vengono completatti
  nell'unità di tempo
\item
  \textbf{Latenze} ed in particolare
\item
  \textbf{Turnaround Time}: tempo totale tra la sottomissione di un
  processo ed il suo completamento
\item
  \textbf{Response Time}: lasso di tempo che intercorre fra la
  sottomissione di un job ed una risposta (es. stampa a video sul
  terminale) da parte del job
\item
  \textbf{Fairness}: equità del tempo di calcolo assegnato a job della
  stessa priorità
\item
  \textbf{Waiting Time}: tempo che un job spende inattivo nella propria
  coda
\end{itemize}

In pratica è non è possibile ottimizzare tutti questi parametri
contemporaneamente (si consideri ad esempio il throughput e la latenza)
ed è necessario realizzare un compromesso. I criteri con cui si realizza
quest'ultimo sono caratteristici del particolare algoritmo di scheduling
e rispondono a particolari esigenze, ad esempio, nel caso di un sistema
in cui la sottomissione di job è un servizio a pagamento, parametri come
la fairness ed il waiting time diventano prioritarì.

Alcuni esempi algoritmi di scheduling sono Unix Scheduling, FCFS (first
Come, first Served), SJR (Shortest Job first), Small Job first / Big Job
first, Priority Scheduling, Round Robin, Multilevel Queue, Multilevel
Feedback Queue, FairShare.

\section{Principi di sicurezza}\label{principi-di-sicurezza}

Write here\ldots{}

\section{Grid Computing}\label{grid-computing}

Write here\ldots{}

\section{Cloud Computing \& Storage}\label{cloud-computing-storage}

Le tecnologie di cloud computig hanno avuto negli ultimi anni largo
impiego nelle realtà aziendali ed è sempre più probabile che nel futuro
diventeranno parte integrante dellle infrastrutture per il calcolo
scientifico (attualmente già in fase di sperimentazione).

Storicamente Amazon è stata fra le prime aziende a sperimentare queste
tecnologie ed a offrire servizi collegati. Ciò nacque dalla esigenza di
sfruttare, dunque vendere, le risorse di calcolo in eccesso a propria
disposizione, le quali erano state acquistate per affrontare i picchi di
domanda sui propri store online in certi periodi dell'anno.

La soluzione ideata fu di mettere a disposizione queste risorse di
calcolo ad altre aziende che avessero simili problemi di picchi di
carico. Naturalmente una soluzione di questo genere comportò l'ideazione
di tecnologie atte a rendere utilmente disponibili queste risorse ed
isolare fra loro gli utenti di queste ultime. Oggi, dopo alcuni anni di
sviluppo, queste tecnologie vengono complessivamente indicate come
\textbf{cloud computing}.

Nel caso del cloud computing la \emph{tecnologia abilitante}, ovvero
senza la quale questi sviluppi non sarebbero stati possibili, è stata la
\textbf{virtualizzazione}.

\subsection{Virtualizzazione}\label{virtualizzazione}

Informalmente, con virtualizzazione si intende la creazione di una
versione virtuale di qualcosa, che può essere ad esempio una piattaforma
hardware, un dispositivo o un sistema operativo. Questo comporta la
creazione di un livello di astrazione intermedio che riproduce certe
funzionalità e nasconde i dettagli di ciò che avviene a più basso
livello, interfacciandosi con i livelli di astrazione superiore come
farebbe l'oggetto che è stato virtualizzato.

Un esempio di questo genere di tecnologie è la \textbf{Java Virtual
Machine} (JVM), ovvero un processore virtuale in grado di eseguire un
set di istruzioni detto \textbf{Java bytecode}. In questo caso se si
vuole eseguire codice Java, quest'ultimo deve essere compilato in
bytecode ed eseguito su JVM. Questo risolve (in teoria) non solo il
problema della portabilità del codice, che in questo modo è indipendente
dalla piattaforma, ma anche dello stesso eseguibile, ovvero il java
bytecode che può essere eseguito su architetture e sistemi operativi
diversi purchè sia stata implementata per questi ultimi una JVM.

Nel discorso che segue per virtualizzazione si intenderà
\emph{virtualizzazione dell'intera piattaforma hardware}, ovvero dove
generalmente il livello di astrazione creato si frappone fra l'hardware
fisico ed il sistema operativo.

\subsubsection{Tipi di virtualizzazione}\label{tipi-di-virtualizzazione}

Si possono distinguere diversi tipi di virtualizzazione ed il primo che
è possibile evidenziare è la \textbf{emulazione}. Si parla di emulazione
quando i sistemi virtuali possiedono una architettura diversa
dall'ospite ed usano un differente set di istruzioni. In altri termini
nella emulazione l'hardware virtuale viene simulato (a spese di maggiori
risorse, dunque \textbf{overhead}).

Il caso in cui i sistemi virtualizzati (\emph{guest}) abbiano la stessa
architettura dei sistemi ospite (\emph{host}) si può distinguere ancora
in \textbf{virtualizzazione completa} e \textbf{paravirtualizzazione}.

Nel caso della virtualizzzazione completa il sistema operativo ospite
non è a conoscenza di essere eseguito su di una macchina virtuale e vede
le stesse interfacce di una macchina fisica e dunque funziona senza
alcuna modifica. In questo caso il livello software fra le macchine
virtuali (VM) ed la macchina fisica viene detto \textbf{hypervisor} ed
ha l'onere di tradurre (o inoltrare) le \emph{system call} delle prime
verso quest'ultima.

Nel caso della paravirtualizzazione i sistemi operativi delle VM sono a
conoscenza della virtualizzazione dell'hardware sottostante e possono
pertanto essere messe in atto delle ottimizzazioni. Queste consistono in
modifiche della interfaccia, tramite chiamate speciali dette
\textbf{hypercall}, permettendo ad esempio di eseguire istanze di codice
in un ambiente non virtuale, dunque direttamente sull'hardware.

Gli hypervisor si distinguono a loro volta fra

\begin{itemize}
\tightlist
\item
  Tipo 1 o \textbf{bare metal}
\item
  Tipo 2 o \textbf{hosted}
\end{itemize}

Gli hypervisor bare metal vengono eseguiti direttamente sull'hardware e
le funzionalità di virtualizzazione si fondono insieme ad un kernel
specifico in un SO \emph{leggero} (dedicato alla virtualizzazione, non
general purpose) che include anche tutti i driver per le periferiche ed
il sistema di gestione/realizzazione delle macchine virtuali. Sono
esempi di hypervisor bare metal: XEN, VMware vSphere, Parallels Server
4, Bare Metal e Hyper-V.

Gli hypervisor hosted invece sono normali applicativi eseguiti
all'interno di un sistema operativo, ed in particolare installati ed
eseguiti in user space, in grado di fornire funzionalità di
virtualizzazione. Un esempio di questi ultimi è Oracle VirtualBox.

Esiste in realtà un terzo genere di hypervisor, in qualche misura ibrido
rispetto ai primi due, detto \textbf{KVM} (*Linux Kernel-based Virtual
Machine**), il quale è una infrastruttura di virtualizzazione che
trasforma il kernel Linux in un hypervisor. In questo caso le funzioni
di virtualizzazione vengono offerte da applicativi in user space che
tuttavia fanno uso di interfacce esposte da un modulo del kernel Linux.
Inoltre KVM può approfittare di alcune estensioni hardware dedicate alla
virtualizzazione, in particolare legate alla gestione della memoria, ed
in genere supportate dalle ultime generazioni di processori che
permettono un migliore sfruttamento delle risorse (riduzione overhead).

\subsubsection{Vantaggi della
virtualizzazione}\label{vantaggi-della-virtualizzazione}

L'avanzamento tecnologico degli ultimi anni ha prodotto una situazione
in cui (generalmente ed in media) gli applicativi non sono più in grado
di saturare le risorse hardware disponibili, sia nel caso di personal
computing che, in maniera ancora più accentuata, in ambito aziendale.

Infatti si stima che un moderno server venga sfruttato solo al 15-20\% e
pertanto la virtualizzazione offre, permettendo di ospitare 3 o 4
machine virtuali sullo stesso hardware, il vantaggio di un miglior
sfruttamento delle risorse, chiamato in ambito aziendale
\textbf{consolidamento dei server}, e dunque la \textbf{riduzione dei
server fisici}. Quest'ultima comporta la riduzione dei consumi
energetici (quindi la necessità di impianti di raffreddamento meno
potenti), dei \textbf{guasti hardware}, dei tempi tecnici per il
montaggio ed il cablaggio, del numero di armadi (\emph{rack}) e pertanto
l'abbattimento dello spazio dedicato in sala macchine per questi ultimi
ed il loro relativo cablaggio.

Inoltre il software, ed in particolar modo il sistema operativo in
esecuzione su una macchina, è in genere strettamente legato all'hardware
su cui viene eseguito. Pertanto se, ad esempio per un guasto hardware,
si deve migrare una installazione da una macchina ad un'altra, si dovrà
spendere del tempo nella configurazione del nuovo hardware. A questo
proposito la virtualizzazione offre il vantaggio della
\textbf{indipendenza hardware}, infatti il sistema operativo guest non
si interfaccia con l'hardware fisico ma con un livello di astrazione di
quest'ultimo e l'amministratore potrebbe spostare o clonare una macchina
virtuale su altre macchine fisiche che eseguano lo stesso ambiente di
virtualizzazione senza ulteriore lavoro (se si esclude la configurazione
di quest'ultimo).

L'indipendenza dallo hardware in alcuni casi non rappresenta
semplicemente una semplificazione della amministrazione straordinaria di
sistema ma una necessità. Il tipico esempio è il supporto di vecchie
applicazioni (\textbf{legacy}), ad esempio sviluppate per DOS, non in
grado di supportare hardware più recente. In questi casi gli ambienti di
virtualizzazione permettono l'esecuzione anche di sistemi \emph{legacy}
permettendo ai responsabili IT di liberarsi del vecchio hardware non più
supportato e più soggetto a guasti.

Un ulteriore vantaggio è la \textbf{standardadizzazione del runtime},
ovvero mettere appunti ambienti di sviluppo (ma anche postazioni di
lavoro, server di posta, etc.) omogenei in maniera semplicemente
riproducibile, e la \textbf{creazione di ambienti di test}, ovvero di
ambienti isolati che possano essere facilmente creati, distrutti o
ripristinati ad uno stato precedente, per testare software.

\subsubsection{Svantaggi della
virtualizzazione}\label{svantaggi-della-virtualizzazione}

Una delle caratteristiche richieste fin da principio alle tecnologie di
virtualizzazione è l'isolamento dei dati e dei processi dei sistemi
virtualizzati. In particolare questo aspetto, come si vedrà in seguito,
emerge preponderantemente nei casi in cui gli utenti dei sistemi
virtualizzati acquistano come servizio l'infrastruttura di
virtualizzazione da terzi e pertanto si richiede che sullo stesso
hardware coesistano diversi utenti senza collidere.

Tuttavia l'isolamento è realizzato dal software di virtualizzazione ed i
\textbf{problemi di sicurezza} che derivano da bachi di quest'ultimo
sono fra i principali svantaggi di queste tecnologie.

Inoltre l'introduzione di un livello software fra i sistemi
virtualizzati e lo hardware ha come inevitabile contropartita una
\textbf{diminuzione delle performance} (seppure minima, grazie al
perfezionamento di queste tecnologie ed al loro elevato grado di
maturità). Concretamente questa riduzione delle perfomance consiste in
un overhead di esecuzione praticamente non rilevabile ed in una
\textbf{riduzione del throughput di I/O} su disco e di rete misurabili
(meno importanti nel caso di paravirtualizzazione e trascurabili per
quest'ultimo per traffico di rete).

\subsection{Cloud Computing}\label{cloud-computing}

Le peculiarità delle tecnologie di virtualizzazione hanno introdotto
negli ultimi anni una modifica radicale degli approcci al calcolo, con
applicazioni al calcolo scientifico.

Storicamente in principio, in ambito aziendale ma anche scientifico, il
calcolo era affidato ai \emph{mainframe}, grandi macchine con differenze
architetturali (sia hw che sw) sostanziali rispetto ai server
comunemente diffusi oggi, le cui risorse, ovvero tempo di calcolo,
venivano allocate in maniera molto critica -- sistemi di questo genere
continuano ad esistere ad esempio in banche o agenzie governative.

Successivamente, grazie agli sviluppi tecnologici ed in particolare la
miniaturizzazione dell'elettronica, si affermò un modello di calcolo
basato su hardware in pieno controllo dell'utente (\emph{personal
computers}), successivamente sull'accesso a risorse di calcolo remote
(\emph{client-server computing}) ed in particolare (successivamente)
sull'accesso a documenti remoti (\emph{Web}).

Negli ultmi dieci anni, in particolare grazie alle tecnologie di
virtualizzazione, si è sviluppato un nuovo paradigma basato sull'accesso
a risorse di calcolo remote con caratteristiche innovative detto
\textbf{cloud computing}.

\subsubsection{Traditional infrastructure
model}\label{traditional-infrastructure-model}

Sia in ambito aziendale che scientifico la crescita nel tempo di una
istituzione in genere comporta una maggiore richiesta di risorse di
calcolo, ad esempio per un aumento della base di utenti.

L'approccio tradizionale a questo problema era di acquistare un surplus
di risorse, rispetto alla domanda attuale, per tenere il passo delle
delle richieste future. Le previsioni sul tasso di crescita e la
frequenza degli aggiornamenti determinavano l'entità di questo surplus.
Si osserva che, anche nel caso ideale di una crescita monotona con
previsioni affidabili sul tasso di crescita, questa è una soluzione non
ottima in quanto per certi lassi di tempo vengono allocate (dunque
pagate) risorse che rimangono inutilizzate.

Nei casi reali questo genere di soluzioni presenta ulteriori svantaggi.
Infatti in questi ultimi la domanda di risorse, pur potendo avere in
media semplice (ad esempio una crescita lineare), è in genere soggetta a
fasi di crescita e contrazione, con un tasso non prevedibile. Le
conseguenze di queste oscillazioni sono dei \textbf{surplus con perdite
non accettabili} (spese troppo ingenti rispetto al consumo di risorse) e
\textbf{periodi di deficit}.

Da questa discussione emerge che i problemi di questo approccio sono la
lentezza degli interventi di aggiornamento delle infrastrutture e la
loro irreversibilità, nel senso che una volta acquistate non è possibile
(o non è conveniente) cederle o smantellarle e rappresentano un costo
fisso.

Il \textbf{cloud computin} è un insieme di tecnologie che permette un
approccio al problema delle risorse radicalmente differente, rendendo
possibile ad esempio approntare una infrastruttura di calcolo in tempi
dell'ordine di alcuni minuti, contro le diverse settimane necessarie nel
caso di IT tradizionale.

\subsubsection{Definizione di cloud
computing}\label{definizione-di-cloud-computing}

In letteratura la definizione di riferimento per le tecnologie di cloud
computing è quella data dal \emph{National Institute of Standards and
Technology} USA (\textbf{NIST}), che in sintesi definisce queste ultime
come

\begin{quote}
\textbf{fornitura} di tecnologia di \textbf{informazione e
comunicazione} (ICT) come \textbf{servizio}
\end{quote}

Si osserva che concettualmente il termine di maggior peso in questa
definizione è \emph{servizio}, il quale racchiude il contenuto
innovativo di queste tecnologie (e dei paradigmi che rendono possibili).

La definizione del NIST inoltre prevede alcuni punti

\begin{itemize}
\tightlist
\item
  \textbf{On demand self service:} l'utente del servizio deve essere in
  grado di ottenere in maniera semplice, trasparente ed automatica
  risorse di calcolo, come ad esempio CPU time o dischi, alla bisogna,
  senza la necessità di una interazione diretta con gli amministratori
  dei service provider o di un intervento umano.
\item
  \textbf{Broad network access:} il servizio deve essere distribuito ed
  accessibile tramite la rete
\item
  \textbf{Resource pooling:} le risorse di calcolo, fisiche o virtuali,
  disponibili al service provider devono essere distribuite e
  riassegnate dinamicamente agli utenti in base alla domanda senza che
  questi ultimi collidano, ovvero realizzando una condizione di
  \textbf{isolamento} dei dati e dei processi di questi ultimi in un
  ambiente fortemente \textbf{multi-tenant}.
\item
  \textbf{Rapid elasticity:} il servizio deve essere non solo in grado
  di fornire risorse di calcolo \emph{elasticamente}, ovvero in base al
  carico, ma specialmente in maniera \textbf{rapida}, ovvero adattandosi
  a quest'ultimo in tempi brevi ed idealmente tenendo il passo della
  domanda in tempo reale.
\item
  \textbf{Measured service:} i service provider devono essere in grado
  di misurare l'erogazione del servizio, utilizzando metriche di un
  livello di astrazione adeguato al genere di servizio, in modo da poter
  sia addebitare eventuali costi di servizio che, più in generale,
  ottimizzare le risorse fra gli utenti, ad esempio in base alle
  priorità assegnate alle organizzazioni a cui appartengono.
\end{itemize}

Una analogia utile a cogliere alcuni aspetti del cloud computing è
quella con il noleggio auto:

\begin{itemize}
\tightlist
\item
  l'utente effettua una prenotazione telefonica o online senza la
  necessità di un intervento umano (\emph{self service}) e accendendo un
  contratto temporaneo relativo alla prestazione particolare (\emph{on
  demand})
\item
  il service provider (agenzia di autonoleggio) possiede una rete di
  distribuzione geograficamente distribuita (\emph{broad network
  access})
\item
  le risorse (il parco vetture) viene riorganizzato per garantire il
  servizio in maniera trasparente all'utente (\emph{resource pooling})
\item
  i dettagli del noleggio sono stabiliti in base alla domanda e possono
  essere semplicemente modificati (\emph{rapid elasticity})
\item
  l'addebito agli utenti viene effettuato in base al consumo
  (\emph{measured service})
\end{itemize}

\subsubsection{Service models}\label{service-models}

Il paradigma reso possibile dalle tecnologie di cloud computing presenta
delle analogie con altri sistemi di condivisione di risorse di calcolo
geograficamente distribuite, come ad esempio la GRID, tuttavia \emph{nel
cloud computing il focus è sul servizio} e questa, che è una
caratteristica distintiva di queste tecnologie, ne ha sancito il
successo in ambito aziendale.

Nel caso del grid computing la risorsa messa a disposizione dalla
infrastruttura è essenzialmente la capacità di eseguire un applicativo,
ovvero di sottomettere un job ad un \textbf{batch system} che può
amministrare risorse facendo uso di tecnologie vicine a quelle adoperate
per cloud computing. Quest'ultima è una differenza sostanziale rispetto
al cloud computing propriamente detto in cui il panorama dei servizi
offerti è molto più ricco e variegato, comportando un profondo divario
nella implementazione di queste due tecnologie oltre che nell'approccio
al problema.

Inoltre il grid computing si è affermato solamente all'interno della
comunità scientifica mentre il cloud computing ha suscitato grande
interesse nelle realtà aziendali ed è oggi una tecnologia matura
ampliamente adottata in questi contesti, con alcune sperimentazioni in
ambito scientifico.

Le tecnologie di cloud computing vengono in genere classificate in base
alla tipologia di servizi che vengono erogati e si distinguono, in prima
istanza, in tre livelli

\begin{itemize}
\tightlist
\item
  Infrastructure as a Service (\textbf{IaaS})
\item
  Platform as a Service (\textbf{PaaS})
\item
  Software as a Service (\textbf{SaaS})
\end{itemize}

e successivamente il concetto di \emph{Something as a Service} è stato
esteso ad altri sottolivelli e servizi specifici.

\subsubsection{Infrastructure as a
Service}\label{infrastructure-as-a-service}

Nel caso di IaaS il service provider mette a disposizione essenzialmente
un ambiente di virtualizzazione e fornisce un certo numero di macchine
virtuali, con certe caratteristiche, in base alla domanda dell'utenza.
Pertanto in questo livello di servizio tutto ciò che arriva fino al
livello del sistema operativo è di competenza del service provider,
mentre ciò che viene installato sulla macchina (middleware, software,
etc.) diventa responsabilità dell'utente.

Semplificando, la differenza rispetto alla virtualizzazione su hardware
in proprio controllo è che è possibile ottenere in pochi passaggi e su
richiesta una macchina virtuale, con CPU, memoria e dischi richiesti,
con un sistema operativo scelto senza dover conoscere i dettagli, o
dover spendere energie nell'implementazione, della infrastruttura
sottostante.

In questo caso l'utente può configurare il servizio assemblando
virtualmente macchine, dischi, componenti di rete, etc, in maniera
automatizzata senza interaggire direttamente con gli amministratori del
datacenter in cui è fisicamente ospitato lo hardware (ad esempio tramite
una interfaccia Web). Tuttavia questo da solo non sarebbe sufficiente a
definire questo genere di servizi cloud computing, infatti ciò che
permette di definire quest'ultimo cloud computing è la
\textbf{flessibilità} del servizio, in accordo al concetto più generale
di \emph{rapid elasticity} (che diventà \textbf{dinamicità} o
\textbf{scalabilità dinamica} del caso SaaS o PaaS).

Ad esempio un ipotetico servizio in cui sia possibile noleggiare una
macchina virtuale con certe caratteristiche accendendo un contratto
specifico per quella soluzione (\emph{managed hosting}) non potrebbe
essere definito cloud computing: nel caso di IaaS è importante che
l'utente sia in grado di modificare le caratteristiche del servizio,
ovvero l'infrastruttura e dunque CPU, dischi, etc., dinamicamente.

Questi servizi sono caratterizzati da:

\begin{itemize}
\tightlist
\item
  Ambienti multitenant virtualizzati e sistemi critici per l'isolamento
  dei dati e dei processi degli utenti
\item
  Addebito in genere stabilito in base al wall time, non in base all'uso
\item
  Nelle implementazioni reali, possibilità di accoppiare servizi per
  l'installazione e la manutenzione dei sistemi operativi o del runtime
\item
  Accesso con diritti di amministrazione da parte dell'utente
\end{itemize}

Lo svantaggio peculiare della IaaS consiste nella lentezza e nella
complessità nella creazione o modifica di macchine virtuali. In risposta
a queste difficoltà molti utenti hanno preferito sistemi più evoluti,
nella direzione di PaaS e SaaS, in cui i sistemi scalino dinamicamente,
in maniera più automatica e trasparente all'utente, e IaaS è rimasto un
servizio riservato ad utenti con esigenze piuttosto specifiche.

\subsubsection{Platform as a Service}\label{platform-as-a-service}

Nel caso Platform as a Service quello che fornisce il service provider è
un sistema dove il runtime è già disponibile, ovvero dove l'applicazione
oggetto del servizio è pronta per essere usata dallo sviluppatore, che
in questo contesto è l'utente del servizio. Ad esempio prima di poter
sviluppare codice in C è necessario istallare un compilatore, un
debugger ed in genere delle librerie, nel caso di PaaS allo sviluppatore
viene fornito un ambiente pronto per la compilazione e l'esecuzione del
codice, senza che quest'ultimo sia a conoscenza dei dettagli
dell'infrastruttura che rende questo possibile o dell'onere di doverla
preparare.

Nel caso PaaS il service provider fornisce all'utente tutti gli
strumenti per lo sviluppo remoto delle sue applicazioni in maniera
semplice (ad esempio tramite l'uso di interfacce Web) e la possibilità
di gestire in maniera semplice l'intero ciclo di vita di queste ultime.

Sinteticamente Platform as a Service è un metodo per l'erogazione di
risorse di calcolo attraverso una piattaforma, ovvero tramite una
infrastruttura di componenti hardware e software disponibile all'uso e
di cui non è necessario conoscere i meccanismi interni. Tutto ciò
permette in questo modo di abbattere i tempi ed i costi per lo sviluppo
ed il collaudo di applicazioni ed ha segnato il successo di questo
approccio. Tuttavia questo da solo non sarebbe sufficiente per parlare
di questi servizi come cloud computing, infatti la caratteristica
principale di questi servizi è di scalare dinamicamente in base alle
richieste degli applicativi sottomessi.

Questo tipo di servizi è caratterizzato da:

\begin{itemize}
\tightlist
\item
  Ambienti multitenant
\item
  Scalabilità dei sistemi
\end{itemize}

Il principale svantaggio per l'utenza nella sottoscrizione a questi
servizi è la dipendenza degli applicativi dalla piattaforma particolare
e dunque la difficoltà, o addirittura l'impossibilità, di migrare verso
un'altro service provider in un secondo momento.

Come già osservato i service provider di IaaS hanno gradualmente
inglobato servizi di PaaS ed oggi è molto difficile trovare un service
provider che faccia esclusivamente PaaS.

\subsubsection{Software as a Service}\label{software-as-a-service}

L'ultimo livello nella scala dei servizi offerti dal service provider è
il caso \textbf{Software as a Service}, o \textbf{cloud application}. In
questo caso l'utente del servizio è effettivamente l'utente finale ed a
quest'ultimo viene fornito direttamente l'applicativo. Alcuni esempi
noti sono \textbf{Gmail} e \textbf{Google Documents}.

Questi applicativi sono probabilmente la forma più popolare di cloud
computing e sono in genere di utilizzo molto immediato, infatti sono in
genere accessibili direttamente tramite un browser web e rimuovono la
necessità di installazione e configuratione del software su computer
individuali. Comportano inoltre notevoli semplificazioni anche per i
fornitori del servizio, specialmente nell'erogazione del supporto dal
momento che tutta l'infrastruttura (applicativi, rutime, sistema
operativo e hardware) sono loro direttamente accessibili.

Sinteticamente si può dire che SaaS è un metodo per la distribuzione di
applicativi che fornisce un accesso remoto alle funzionalità di questi
ultimi, usualmente traimte interfacce Web, a diversi utenti forniti di
licenza. Questi servizi sono in genere caratterizzati da

\begin{itemize}
\tightlist
\item
  Ambienti multitenant e sistemi per l'isolamento dei dati degli utenti
\item
  Universalmente accessibili, ovvero entro certi limiti indipendenti dal
  software o hardware particolare dell'utenza
\item
  Addebiti stabiliti in base all'uso
\item
  Grande scalabilità dei sistemi per l'erogazione del servizio
  (intrinseca nella gestione del servizio e completamente delegata al
  fornitore)
\item
  Sistemi di gestione delle licenze quindi, termini più astratti,
  gestione critica dei permessi
\item
  Isolamento dei dati degli utenti
\end{itemize}

I principali problemi per l'utenza di questi servizi sono legati
all'accesso, la persistenza ed al possesso ai dati, con ovvie
implicazioni per la privacy degli utenti. Infatti \textbf{il servizio ed
i dati sono strettamente collegati}, cosa che comporta notevoli
conseguenze come ad esempio la possibile perdita dei dati in caso di
interruzione del servizio.

\subsubsection{Alcuni esempi di service
models}\label{alcuni-esempi-di-service-models}

Queste definizioni sono piuttosto astratte e generali, è possibile
vedere con alcuni esempi concreti la loro declinazione nel mondo reale.

\paragraph{SaaS case study: Salesforce.com e Google
Apps}\label{saas-case-study-salesforce.com-e-google-apps}

Un importante esempio di Software as a Service è
\href{www.salesforce.com}{salesforce.com} che fornisce software per
gestire le relazioni con i clienti per le aziende
(\emph{business-to-business}). In questo caso l'utente, ovvero
l'azienda, compone le proprie applicazioni assemblando pacchetti
relativi a particolari servizi (modulo gestione clienti, modulo gestione
fornitori, gestione del magazzino, etc.).

Nella esperienza comune Google Apps rappresenta un esempio classico di
SaaS rivolto agli utenti finali, ovvero i consumatori. Tuttavia questi
servizi (oltre a molti altri) sono rivolti anche alle aziende, alle
quali viene vengono offerti in una forma simile a quella rivolta ai
consumatori ma con la possibilità di personalizzazioni.

Non solo, infatti alcuni di questi servizi vengono offerti gratuitamente
ad enti pubblici o di rilievo in cambio della pubblicità derivante dalla
adesione ai servizi (ed alla raccolta di dati).

Si osserva comunque che Google è in realtà fornisce PaaS, infatti le
applicazioni offerte all'utenza condividono le stesse tecnologie, che
vengono messe a disposizione degli sviluppatori: questi toolkit
incarnano il concetto di PaaS.

Si osserva come questo esempio metta bene in luce le limitazioni
dell'approccio PaaS per gli sviluppatori, infatti un ipotetico
applicativo che faccia uso dei servizi di geolocalizzazione di Google
diventa dipendente da quest'ultimo.

\paragraph{IaaS e PaaS case study: Amazon AWS e Windows
Azure}\label{iaas-e-paas-case-study-amazon-aws-e-windows-azure}

Il caso di studio di Amazon AWS è tra i più significativi in quanto la
gamma dei servizi offerti è così estesa da poter illustrare tutti gli
aspetti di IaaS e PaaS.

Storicamente il primo dei servizi offerti da Amazon Web Services è stato
\emph{Amazon Simple Storage Service} (\textbf{S3}), ovvero un sistema
per la scrittura e la lettura di dati diverso dal semplice storage
remoto (2006). Infatti questo servizio possedeva delle peculiarità che
lo rendevano un vero esempio di PaaS. Ad esempio l'accesso ad i dati
avveniva programmaticamente: ovvero all'interno di un linguaggio di
programmazione tramite delle apposite librerie.

Un'altra caratteristica innovativa di S3 dispobilile fin dalla nascita
del servizio era quella di richiedere che i dati venissero conservati in
data center geograficamente distribuiti.

In un secondo momento è stato varate un servizio per la creazione di
macchine virtuali, ovvero \emph{Amazon Elastic Computing Cloud}
(\textbf{EC2}), che rappresenta la formulazione classica di IaaS.

\textbf{Windows Azure} (in precedenza \textbf{Microsoft Azure}) è
un'altro esempio di di PaaS e IaaS. Quest'ultimo si differenzia rispetto
ad AWS nel fatto di essere fortemente orientato allo sviluppo di nuove
applicazioni nativamente in cloud.

\subsubsection{Deployment and isolation
models}\label{deployment-and-isolation-models}

Quanto presentato fin'ora potrebbe indurre a credere che i casi d'uso
delle tecnologie di cloud computing riguardino solamente alcune grandi
aziende (Amazon, Google, etc.), tuttavia non è questo il caso.

Infatti cloud computing è appunto una tecnologia ed i suoi impiegi
possono essere classificati in base a 3 caratteristiche indipendenti:
\emph{service model}, \emph{deployment model} e \emph{isolation model}.

I servizi degli esempi precedenti (EC2, Windows Azure,\ldots{}) sono
infrastrutture di cloud che secondo lo standard viengono definite
pubbliche, nel senso che chiunque può accedere a questi ultimi dietro
compenso economico.

La caso opposto è quella di qualcuno che costruisca una infrastruttura
di questo tipo per se stesso e quindi permetta l'accesso a certi utenti
particolari. Benchè controintuitivo, questo genere di soluzioni viene
scelto dalla gran parte delle istituzioni sia in ambito accademico che
aziendale: ovvero queste istituzioni utilizzano tecnologie di cloud
computing su risorse che sono completamente sotto il loro controllo.

In generale si distinguono diversi modelli di erogazione dei servizi:

\begin{enumerate}
\def\labelenumi{\arabic{enumi}.}
\tightlist
\item
  \textbf{Public cloud:} L'utente ha diritto all'accesso di risorse di
  calcolo remote che sono allocate dinamicamente su richiesta da
  quest'ultimo, attraverso applicazioni web o API, via interntet a
  seguito della accensione di un contratto. Il service provider addebita
  all'utente una somma calcolata in base all'utilizzo delle risorse.
\item
  \textbf{Private cloud:} L'utente accede a risorse di calcolo remote
  che sono fornite da una istituzione di cui fa parte. Spesso l'adozione
  di queste tecnologie è il primo passaggio prima di una transizione a
  modelli di Public Cloud. 
\item
  \textbf{Hybrid cloud:} L'utente ha accesso a risorse sia locali che
  remote e l'accesso a risorse remote è implementato tramite tecnologie
  di cloud computing. In genere queste soluzioni sfruttano le tecnologie
  di cloud computing per funzioni specifiche o in situazioni di carico
  di lavoro straordinario. Un caso ricorrente, sia in ambito aziendale
  che accademico, è di fare affidamento a risorse di cloud computing
  pubbliche in determinati periodi dell'anno ed un esempio in fisica
  delle alte energie è l'uso di AWS da parte di CMS. In questo modo è
  stato possibile affrontare richieste di calcolo particolarmente
  intensive e superiori alle capacità dell'esperimento, acquistando
  all'asta istanze di calcolo EC2 (\textbf{Spot Istances} e \textbf{Spot
  Fleet}), in quanto per Amazon i periodi di inattività rappresentano un
  costo senza profitto.
\item
  \textbf{Community cloud:} L'utente ha accesso a risorse remote messe a
  disposizione da una organizzazione di diverse istituzioni le quali
  collaborativamente condividono una infrastruttura comune di cloud
  computing. Rispetto al Public cloud le differenze consistono nel minor
  numero di utenti e nel sistema di addebito del servizio, invece
  rispetto al Private cloud è possibile partizionare i costi di
  implementazione e manutenzione del servizio fra le diverse
  istituzioni.
\end{enumerate}

Per isolamento si intendono una serie di aspetti come: -
\textbf{Segmentazione delle risorse:} gli utenti devono avere pieno
accesso alla quantità di risorse che gli è stata destinata e non altre -
\textbf{Protezione dei dati:} impedire l'accesso o la scrittura di dati
da parte di utenti senza permessi - \textbf{Sicurezza delle
applicazioni:} impedire l'esecuzione o la modifica di applicazioni da
parte di utenti senza permessi - \textbf{Auditing:} monitoraggio degli
accessi ai propri file o applicativi da parte degli utenti

ed i modelli di isolamento sono essenzialmente due:

\begin{enumerate}
\def\labelenumi{\arabic{enumi}.}
\tightlist
\item
  Infrastrutture dedicate, con singolo utente o diversi utenti
  appartenenti alla stessa istituzione
\item
  Infrastrutture multitenant, ovvero con diversi tipi di utenti per cui
  l'isolamento è critico
\end{enumerate}

\subsubsection{Considerazioni}\label{considerazioni}

\paragraph{Interesse nei confronti del Cloud
Computing}\label{interesse-nei-confronti-del-cloud-computing}

L'interesse nei confronti di una nuova tecnologie in genere segue un
andamento che si divide in diverse fasi

\begin{enumerate}
\def\labelenumi{\arabic{enumi}.}
\tightlist
\item
  Nascita della tecnologia e repentina crescita dell'interesse
\item
  Picco di aspettative e massima attenzione nei confronti di
  quest'ultima
\item
  Rapida decrescita e picco negativo dell'interesse
\item
  Maturazione della tecnologia con una più graduale crescita
  dell'interesse
\item
  Plateau di produttività caratterizzato da un interesse stabile nel
  tempo
\end{enumerate}

La visibilità del cloud computing ha seguito questo andamento generale
ed attualmente ci troviamo nel plateau di produttività.

\paragraph{IaaS Cloud Stack più
usati}\label{iaas-cloud-stack-piuxf9-usati}

Allo stato attuale esistono diversi software (toolkit) detti
\textbf{IaaS Cloud Stack} per realizzare una infrastruttura di cloud
computing di tipo IaaS. Alcuni di questi sono proprietari, come
\textbf{WMware} e \textbf{Microsoft Hyper-V}, mentre altri sono
\emph{open source}, \textbf{OpenStack}, \textbf{Apache CloudStack} ed
\textbf{OpenNebula}, ma in ogni caso si tratta di software che può
essere installato e configurato su hardware di proprio controllo per
implementare un cloud privato (ma che può eventualmente essere reso
successivamente publico).

A differenza di quest'ultimo il software impiegato da piattaforme di
cloud pubblico, come ad esempio AWS, non può essere impiegato al difuori
della rispettiva piattaforma ed è accessibile solo tramite le IPA
pubbliche.

\paragraph{Avvantaggiarsi delle tecnologie
cloud}\label{avvantaggiarsi-delle-tecnologie-cloud}

Deve essere chiaro che l'utilizzo delle tecnologie cloud, ad esempio in
campo di calcolo scientifico, non garantisce un vantaggio: infatti
\textbf{è necessario che gli applicativi utilizzati siano sviluppati in
modo da approfittare delle potenzialità offerte da queste tecnologie}.
In particolare questo software deve avere le seguenti caratteristiche

\begin{itemize}
\tightlist
\item
  \textbf{Distribuito:} capacità di eseguire istanze di codice
  parallelamente su macchine con memoria non condivisa
\item
  \textbf{Immutabile:} caratteristica di non modificare (idealmente) lo
  stato della macchina durante la propria esecuzione. Questo permette
  avvantaggiarsi del calcolo distribuito potendo eseguire istanze di
  porzioni di codice su diverse CPU senza che queste debbano comunicare
  tra loro, dunque aggirando le latenze di comunicazione tra i processi.
\item
  \textbf{Failover in app:} capacità intrinseca ed automatica di
  gestione di una eventuale interruzione anomala dei sottoprocessi,
  provvedendo meccanismi per farne partire altri
\item
  \textbf{Scalabilità in app:} capacità di modificare automaticamente e
  dinamicamente la quantità di risorse utilizzate in base al carico ed
  alla disponibilità
\end{itemize}

\paragraph{EGI Federated Cloud}\label{egi-federated-cloud}

La \emph{EGI federated cloud} è una organizzazione che riunisce
istituzioni dotate di sistemi di cloud computing privati ed aggrega
questi sistemi offrendo alla comunità scientifica, in europa e nel
mondo, servizi di tipo IaaS come unico service provider.

Le idee dietro questo progetto internazionale sono essenzialmente
l'adozione di \textbf{standard aperti} per le interfacce esposte
all'utente, in modo da semplificare sia il \emph{resource pooling} che
l'integrazione con gli applicativi degli utenti, e la
\textbf{implementazione eterogenea} delle infrastrutture, lasciando
libertà agli enti aderenti di utilizzare le tecnologie di loro
discrezione dunque semplificando la messa in opera di nuove
infrastrutture o l'integrazione di quelle preesistenti.

Un aspetto interessante di questo \emph{federated cloud} è il
\textbf{catalogo delle macchine virtuali}, ovvero un elenco pubblico di
piattaforme o servizi sviluppati dagli utenti e sottomessi da questi
ultimi. Le macchine virtuali in questo catalogo vengono validate dalla
federazione ed in questo modo è possibile per gli utenti instanziare
servizi sviluppati da altri utenti in un ambiente collaborativo aperto.

\subsection{Cloud Storage}\label{cloud-storage}

Per \textbf{cloud storage} si intende l'utilizzo di tecnologie analoghe
al cloud computing (dunque on demand, self service, network access,
resource pooling, rapid elasticity, measured service) per
l'archiviazione di dati.

A seconda dei casi, queste tecnologie si rendono necessarie

\begin{itemize}
\tightlist
\item
  in relazione al cloud computing, in quanto il problema del
  processamento dei dati non può prescindere da quello
  dell'archiviazione di questi ultimi ed inoltre non è possibile
  utilizzare soluzioni di archiviazione tradizionali in una
  infrastruttura di cloud computing
\item
  per via di necessità specifiche degli utenti, come ad esempio la
  necessità di archiviare grandi moli di dati in maniera semplice e
  trasparente o di doverle condividere.
\end{itemize}

Si individuano tre tipologie di cloud storage nel contesto di servizi di
cloud computing di tipo IaaS, escludendo lo storage volatile dello
spazio disco della istanza di macchine virtuali (al riavvio di una
istanza le modifiche vengono perse)

\begin{itemize}
\tightlist
\item
  \textbf{Posix Storage:} permette, oltre che l'accesso remoto, la
  condivisione di file fra più host. Una semplice implementazione, per
  infrastrutture di piccole dimensioni e non molto complesse, può
  consistere anche di un solo volume NFS (\emph{Network File System})
  montato sulle macchine che devono avere accesso ai file. Tuttavia man
  mano che le infrastrutture scalano in dimensione diventa necessario
  usare file system che possano aumentare dinamicamente le proprie
  dimensioni e migliorare le performance sfruttando parallelamente
  l'hardware a disposizione.
\item
  \textbf{Block Storage:} espone alle macchine virtuali un dispositivo
  di archiviazione virtuale, con una interfaccia equivalente ad un disco
  fisico. Quest'ultima nei sistemi UNIX consiste in un file speciale, in
  particolare in Linux in un \emph{bock device file}, tramite il quale
  le funzionalità dei dispositivi di archiviazione sono accedute tramite
  le chiamate di sistema per la lettura e scrittura di file. In questo
  caso l'utente, proprio come nel caso di un disco fisico, può decidere
  come formattare ed utilizzare il dispositivo. Inoltre in genere
  quest'ultimo è disponibile sulla rete e condiviso da più host
  contemporaneamente e l'utente è in grado di gestire questi dispositivi
  tramite la stessa interfaccia attraverso cui amministra le macchine
  virtuali.
\item
  \textbf{Object Storage:} non permette l'accesso ai dati di tipo Posix
  (open, seek, write, close) ma mette a disposizione una serie di IPA
  per l'accesso e la modifica di questi ultimi. In questo caso non è
  prevista una struttura a directory ma una organizzazione ad
  \textbf{oggetti}, file binari eventualmente provvisti di metadati, e
  \textbf{bucket}, contenitori in cui scrivere/leggere oggetti (ma non
  altri bucket, non è infatti possibile costrurire alberi di bucket ed
  esiste un solo livello). Oltre alle IPA è in genere possibile accedere
  ai file (oggetti) tramite web services sfruttando il protocollo HTTP.
  Un tipico caso d'uso di questa tipologia di cloud storage è la
  memorizzazione delle immagini delle macchine virtuali nei Market Place
  o negli Image Service.
\end{itemize}

\subsubsection{Esempi di tecnologie disponibili per il cloud
storage}\label{esempi-di-tecnologie-disponibili-per-il-cloud-storage}

Nella seguente tabella sono elencati alcuni esempi di tecnologie
disponibili per le varie tipologie di cloud storage

~~~~~~~~~~~~~~~~~~~\textbar{} \textbf{Open Stack} \textbar{}
\textbf{Amazon} \textbar{} \textbf{Others} \textbar{}\\
\textbf{Posix Storage} \textbar{} NA \textbar{} NA \textbar{} GlusterFS,
Lustre, GPFS, CEPH \textbar{}\\
\textbf{Block Storage} \textbar{} Block Storage service (Cinder)
\textbar{} Amazon Elastic Block Store (EBS) \textbar{} CEPH, iSCSI
storage \textbar{}\\
\textbf{Object Storage} \textbar{} Object Storage service (Swift)
\textbar{} S3 \textbar{} CEPH, GlusterFSUFO \textbar{}

mentre nella seguente tabella sono riassunte alcune caratteristiche
delle tipologie di cloud storage per il caso particolare dello stack per
il cloud computing open source \textbf{Open Stack}

\ldots{}

\subsubsection{Caratteristiche di un cloud
storage}\label{caratteristiche-di-un-cloud-storage}

Un servizio di cloud storage possiede le stesse proprietà
caratterizzanti di un servizio di cloud computing, ovvero \emph{on
demand \& self service}, \emph{broad network access}, \emph{resource
pooling}, \emph{rapid elasticity} e \emph{measured service}. Tuttavia
nel caso del cloud storage si aggiungono alcune richieste

\begin{itemize}
\tightlist
\item
  \textbf{High availability:} ovvero la ridondanza dei dati e la
  capacità dei sistemi garantire l'accesso ai dati nonostante la failure
  di un singolo dispositivo hardware.
\item
  \textbf{High Level API:} possibilità di accesso ai dati, oltre che
  eventualmente tramite standard POSIX, anche tramite una interfaccia da
  remoto, standard ed universale (in genere un \emph{web service})
\end{itemize}

In una tecnologia di cloud storage fra gli aspetti di maggior rilievo ci
sono le interfacce, ovvero il modo con cui viene esposto l'accesso ai
file ai livelli di astrazione superiori.

Due esempi diametralmente opposti sono lo standa

\section{Analisi di Big Data}\label{analisi-di-big-data}

Esigenza di utilizzare i big data: cambia l'approccio all'analisi dei
dati: algoritmi più complicati. Si è passati da qualche tera a peta e
adesso anche ad exabyte. Quando si parla di analisi a big data quello
che vogliamo avere è una architettura scalare e non progettata per un
target, ovvero le cui prestazioni richieste dipendano dalla quantità di
dati. Esempio: Facebook il cui numero di dati cresce continuamente del
tempo senza far trasparire un affanno da parte dell'infrastruttura.

\subsection{Apache Hadoop}\label{apache-hadoop}

E un software open source (nato dall'esperienza di Google nei suoi data
center). Vogliamo che sia

\begin{enumerate}
\def\labelenumi{\arabic{enumi}.}
\tightlist
\item
  scalabile
\item
  cost effective
\item
  flessibile
\item
  Fail tolerant
\end{enumerate}

Nel caso specifico la scelta iniziale di hadop è stata quella di partire
da dei moduli di basi e di dare la possibilità a chi serve di
appoggiarsi su quelle librerie di base creare dei progetti correlati.
Alla librerie di base si aggiunge quella del file system e poi due
moduli per gestire anche la parte legata a come far girare dei programmi
sui dati. Quindi in questo caso la gestione dei dati e delle code è
unica e questo porta dei vantaggi. Su questa base (ovvero oggetti per
gestire dati e far girare codice) si sono appaggiati dei progetti
paralleli (come cassandra che è un sistema per creare database).
\subsection{Idea} Nasce da Google. Quando devo far girare una grande
quantità di dati ho due problemi: uno è la CPU e l'altro è leggere i
dati. Questi due problemi sono strettamente correlati. Dobbiamo cercare
di non far stare la CPU senza far niente ma al tempo stesso non dare
troppi dati in modo che non riesce a processarli.

La somma dei dischi deve essere sufficientemente veloce (non è singolo
disco) e anche la rete deve essere sufficientemente veloce. Quest'ultimo
aspetto è in realtà il più costoso. L'idea di google è cambiare come
vengono processati i dati.

FIG 1 sul quaderno.

Il problema dell'approccio innovativo è che se prima l'archiviazione di
dati era dedicata a macchine dedicate, ora non lo è piu e le failure
diventano più importanti e si devono gestire meglio quest'utlime.
Failure: problema ai dischi, problema alla macchina, problema di rete.
Per la prima devo prevedere una ridondanza con delle repliche in ogni
posto. In un datacenter grosso ci sono i domini di fallimento (che sono
problemi che possono capitare tipo switch che si rompono o linee di
alimentazione che si rompono e non fanno funzionare un singolo armadio
ecc).

E' basato su JAVA. Si voleva avere la possibilità di costruire un codice
JAVA che sfruttasse in profondità il sistema. \subsection{Architettura}
Si utilizza un programma JAVA di alto livello che simula l'interfaccia
con il file system tramite Mont point FUSE che simuli il file system.
FUSE è un layer che mi permette di simulare la vista a file system
(cartelle ecc) anche se dietro c'è una cosa più complicata o di più alto
livello.

l'idea di hadoop è quello di risolvere i problemi classici che avvengono
in un data center. Immaginiamo che ci siano tre copie del mio file e ho
due armadi. Conviene che le mie tre copie siano distribuite almeno una
su un altro armadio in modo che se uno si spegne comunque un'altra copia
sta sull'altro.

Chiedo la scrittura ad hadoop specificando la block size e il numero di
repliche. Il client specifica queste cose al namenode che è un server
che controlla tutto. Il namenode controlla dove è disponibile spazio e
scrive il file sul primo datanode che copia poi sull'altro e cosi via.
Questo riduce la complessita del client appoggiando uan parte dei
compiti sul datasistem. Una volta che tutti e tre hanno comunicato al
namenode che hanno scritto la procedura è finita. Quindi la velocità
effettiva è tre volte piu lenta rispetto alla copia singola. Il sistema
quindi ha completato la scrittura di quel blocco e si passa al blocco al
successivo.

Quando il client vuole leggere un file va al namenode e chiede il file.
Il namenode comunica con i datanode in parallelo le richieste sui file
che comunicano i blocchi presenti di quel file. Il vantaggio è che se un
nodo per esempio è occupato o non funziona il sistema può recuperare
dagli altri nodi il dato cercato e questo per l'utente è completamente
trasparente.

Riguardo la placemente startegy lui fa questo quando si richiede di fare
3 copie: una replica in un reck e un altra replica in 2 reck. Questo fa
si che si minimizzano quanto piu possibile l'uso degli uplink di rete
utilizzando la rete con la connessione interna del rack che è meno
costosa. Quindi comunque ho la ridondanza senza aver stressato troppo
l'uplink. Inoltre questi sistemi vengono utilizzati per scrivere poco e
leggerli molte volte. Quidni avere due copie all'interno dello stesso
rack fa si che quando vado a fare analisi potrei leggere al doppio della
velocità utilizzando le due copie all'interno dello stesso rack.

\section{•}

FIG 3

\textbackslash{}end\{document\}
